\section*{Teoretická část}

Budeme pozorovat pohyb harmonického oscilátoru (závaží zavěšené na pružině) ve třech případech: oscilátor kmitá sám a je tlumení je zanedbatelné, oscilátor je tlumený, na tlumený oscilátor působí periodická vnější síla

V případě, že oscilátor není výrazně tlumený, kmitá s kruhovou frekvencí $\omega$.

Působí-li na harmonický oscilátor působit síla úměrná rychlosti pohybu, má jeho pohybová rovnice tvar
\begin{equation} \label{eq::pohybovarovnicetlumeny}
\ddot{y}+2\delta \dot{y}+\omega^2y=0 \,,
\end{equation}
kde $y$ je výchylka tělesa a $\delta$ je tzv. konstanta tlumení.

Pro slabou tlumící sílu ($\delta<\omega$) je řešením rovnice \eqref{eq::pohybovarovnicetlumeny}
\begin{equation}
y= A\cdot e^{-\delta t} \cdot \sin(\omega_1t+\varphi_0) \,,
\end{equation}
kde $A$ a $\varphi_0$ jsou integrační konstanty a $\omega_1$ je kruhová frekvence tlumeného oscilátoru, platí \cite{skripta}
\begin{equation} \label{eq::omega1}
\omega_1^2 = \omega^2 - \delta^2 \,.
\end{equation}

Pokud budeme na oscilátor působit periodickou vnější silou s harmonickým průběhem a kruhovou frekvencí~$\Omega$, bude mít rovnice \eqref{eq::pohybovarovnicetlumeny} tvar
\begin{equation}
\ddot{y} + 2\delta \dot{y} + \omega^2 y = \frac{F_0}{m}\cdot \sin(\Omega t) \,,
\end{equation}
kde $F_0$ je maximální působící síla a $m$ je hmotnost oscilátoru.
Řešení má tvar \cite{skripta}
\begin{equation}
y=A \cdot e^{-\delta t} \cdot \sin (\omega_1 t+ \varphi_0) + A_v \cdot \sin (\Omega t + \gamma) \,,
\end{equation}
kde $A$ a $\varphi_0$ jsou integrační konstanty, $A_v$ a $\gamma$ jsou konstanty: \cite{skripta}
\begin{equation}
A_v=\frac{F_0}{m \omega^2} \frac{1}{\sqrt{
\left( 1- (\Omega/\omega)^2  \right)^2 +
4 (\delta/\omega)^2(\Omega/\omega)^2
}} 
\end{equation}
\begin{equation}
tan (\gamma) =- \frac{2\delta}{\omega}\cdot \frac{\Omega/\omega}{1-(\Omega/\omega)^2} \,.
\end{equation}

Amplutuda $A_v$ nabývá maxima při kruhové frekvenci vnější síly
\begin{equation} \label{eq::resomega}
\Omega_{res}^2 = \omega^2-2\delta^2 \,.
\end{equation}