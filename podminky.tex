\section*{Podmínky, pomůcky a měřící přístroje}
Měření proběhlo při pokojové teplotě (přibližně \SI{25}{\degreeCelsius}) a normílním atmosférickém tlaku.

Jako oscilátor jsme použili permanentní magnet zavěšený na pružině.

POZOR

Pod magnetem byla umístěna cívka připojená k počítači, který nám umožňoval do ní pouštět střídavé napětí o zvolené frekvenci a také zobrazovat časový průběh napětí indukované magnetem.



Indukované napětí by mělo být přibližně přímo úměrné okamžité rychlosti magnetu.
Periodu kmitu můžeme odečíst podle časové vzdálenosti jednotlivých peaků napětí.


Měřili jsme maximální napětí při pohybu magnetu nahoru i dolů.
Při pohybu dolů bylo napětí většinou výrazně vyšší (viz graf \ref{grp::mericipristroje}), takže předpokládáme, že měřící přístroj byl zatížen systematickou chybou.
Při měření konstanty tlumení se tuto chybu pokusíme odstranit tím, že u souboru naměřených peaků napětí nejdříve určíme střed jako průměr nulových a nenulových hodnot.
Dále vezmeme vzdálenost všech bodů od této hodnoty a až 
tento soubor hodnot prokládáme exponenciálou.

\begin{graph}[htbp] 
\centering
%\input{graf.tex}
\caption{Graf 1}
\label{grp::mericipristroje}
\end{graph}

Soustava nebyla kalibrovaná k měření výchylky, takže amplitudy u nuceného kmitání uvádíme pouze v poměru k nejvyšší naměřené hodnotě.
Při pohybu harmonického oscilátoru platí, že maximální výchylka je úměrná maximální rychlosti.
S ohledem na to, že tlumení bylo slabé, aproximujeme pohyb tlumeného oscilátoru netlumeným a budeme uvažovat, že maximální výchylka v krátkém časovém úseku po naměření maximální rychlosti je úměrná této rychlosti a tedy i naměřenému maximu indukovaného napětí.
Při pohybu dolů bylo napětí opět vyšší, relativní velikost výchylky určíme jako rozdíl průměru kladných maxim a průměru záporných maxim (viz graf \ref{grp::nucenyilustrace}).

\begin{graph}[htbp] 
\centering
%\input{graf.tex}
\caption{Graf 1}
\label{grp::nucenyilustrace}
\end{graph}