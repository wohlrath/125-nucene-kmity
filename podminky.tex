\section*{Podmínky, pomůcky a měřící přístroje}
Měření proběhlo při pokojové teplotě (přibližně \SI{25}{\degreeCelsius}) a normálním atmosférickém tlaku.

Jako oscilátor jsme použili permanentní magnet zavěšený na pružině.


Pod magnetem byla umístěna cívka připojená k počítači, který nám umožňoval do ní pouštět střídavé napětí o zvolené frekvenci a také zobrazovat časový průběh napětí indukované magnetem.


Indukované napětí by mělo být přibližně přímo úměrné okamžité rychlosti magnetu.
Periodu kmitu můžeme odečíst podle časové vzdálenosti jednotlivých peaků napětí.


Měřili jsme maximální napětí při pohybu magnetu nahoru i dolů.
Při pohybu dolů bylo napětí většinou výrazně vyšší (viz graf \ref{grp::mericipristroje}), takže předpokládáme, že měřící přístroj byl zatížen systematickou chybou.
Při měření konstanty tlumení se tuto chybu pokusíme odstranit tím, že u souboru naměřených peaků napětí nejdříve určíme střed jako průměr nulových a nenulových hodnot.
Dále vezmeme vzdálenost všech bodů od této hodnoty a až 
tento soubor hodnot prokládáme exponenciálou.

\begin{graph}[htbp] 
\centering
\input{tlumpuvodni.tex}
\caption{Naměřené hodnoty při kmitání tlumeném modrým diskem, graf ilustruje způsob zpracování}
\label{grp::mericipristroje}
\end{graph}

Soustava nebyla kalibrovaná k měření výchylky, takže amplitudy u nuceného kmitání uvádíme pouze v poměru k nejvyšší naměřené hodnotě.
Při pohybu dolů bylo napětí opět vyšší, relativní velikost výchylky určíme jako rozdíl průměru kladných maxim a průměru záporných maxim (viz graf \ref{grp::nucenyilustrace}).

\begin{graph}[htbp] 
\centering
\input{nucenyilustrace.tex}
\caption{Ilustrace metody určení amplitudy při nucených kmitech, poměrnou amplitudu určíme jako vzdálenost dvou čárkovaných přímek}
\label{grp::nucenyilustrace}
\end{graph}